\iffalse 
Chapter 1: Introduction - the topic, the background, why the topic is relevant or of interest to you, what you hoped to achieve, the aims and objectives of the project.  
\fi

\section{Introduction}

\subsection{Online Recommender Systems}

Since their origin in the mid-1990s with systems such as Tapestry \cite{Goldberg92} and GroupLens \cite{Resnick94}, recommender systems have become ubiquitous on the World Wide Web, being employed by some of the worlds largest online businesses as core parts of their offering.

The growth of the Web has given companies the ability to gather unprecedented amounts of data about their users' preferences, both explicitly collected and inferred from their behaviour, while at the same time enabling them to reach users for less cost more often than ever before.

Amazon's system of product recommendations using item-to-item collaborative filtering is regarded as a ``killer feature''\cite{Fortune12}, and is one of the defining features of the Amazon website. The importance of recommendations to Amazon is reflected in their stated mission, ``to delight our customers by allowing them to serendipitously discover great products''\cite{Fortune12}.

Another company which, like Amazon, is synonymous with recommender systems is Netflix, an ``Internet television network'' \cite{NetflixAbout}. In October 2006 Netflix launched ``The Netflix Prize'', a competition with a \$1,000,000 Grand Prize on offer for any team which could beat their own Cinematch recommender system by ``at least 10\%'' accuracy over a fixed set of data \cite{NetflixPrizeRules}. In 2009 the prize was awarded to the BellKor's Pragmatic Chaos team, who had improved on Netflix's own system by 10.06\% \cite{NetflixPrizeCom}. 

\subsection{Aims and Objectives}

My interest in recommender systems is founded in their variety and ubiquity. It occurred to me that I encounter, and am the subject of, dozens of these systems in my everyday life. Whether it's Twitter or Facebook recommending interesting interesting people or content to me, Amazon recommending me a book or film, or even a supermarket targeting special offers to me, I interact with recommender systems all the time. I am fascinated both by how these systems work theoretically and by how they are implementated in practice.

I have kindly been permitted to freely use data from Decanter.com's wine reviews database \cite{DecanterWine} for my project. The sphere of wine recommendations is particularly interesting; wine is at first glance a narrow subject, but it is a nuanced one. Among oenophiles there is an emphasis on personal taste and a strong tradition of rating and grading. 

In this project I aim to build a recommender system based on Decanter.com's wine database, identifying and overcoming the challenges associated with the implementation of a real-world recommender system.

Recommendation quality is important to any recommender system, and I intend to focus strongly on it, but I aim to produce a system satisfactory in both its performance in delivering recommendations and its performance as a web service, with a robust and elegant implementation in code. 

My aim is not to build any kind of graphical interface for the system, but instead to provide a machine readable service API. Where necessary I will also produce batch scripts and command line tools for preparing and manipulating data.

