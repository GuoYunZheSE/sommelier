\iffalse
Chapter 2: Literature Review and Context - the setting of the project in the context of other relevant work or theories or results. How this setting influenced the project.
\fi

\section{Literature Review}\label{literature review}

The term \textit{recommender system} was coined by Resnick and Varian \cite{Resnick97} to describe  a system that ``assists and augments'' the ``natural social process'' of recommendation, with Resnick and Varian stating that they preferred it to the more narrow term ``collaborative filtering'' used by Goldberg et al. \cite{Goldberg92}  to describe their Tapestry system.

The growth of the World Wide Web has seen recommender systems become a ubiquitous part of everyday life, with companies such as Amazon, Facebook, Twitter and Google using making recommendations to millions of us every day.

\ldots

There are several main categories of filtering technique employed in recommender systems. Burke \cite{Burke02} presents five: \textit{collaborative}, \textit{content-based}, \textit{demographic}, \textit{utility-based} and \textit{knowledge-based}. Table \ref{table:burke02} details these methods and their properties.

\begin{table}[ht]
    \caption{Recommendation Techniques, reproduced from Burke, 2002 \cite{Burke02}}
    \centering
    \begin{tabular}{p{2.5cm} p{3.5cm} p{3.5cm} p{3.5cm}}
        Technique & Backgroud & Input & Process
        \\\hline\hline
        Collaborative & Ratings from \textit{U} of items in \textit{I}. & Ratings from \textit{u} of items in \textit{I}. & Identify users in \textit{U} similar to \textit{u}, and extrapolate from their ratings of \textit{i}. \\
        Content-based & Features of items in \textit{I}. & \textit{u}'s ratings of items in \textit{I}. & Generate a classifier that fits \textit{u}'s rating behaviour and use it on \textit{i}. \\ 
        Demographic & Demographic information about \textit{U} and their ratings of items in \textit{I}. & Demographic information about \textit{u}. & Identify users that are demographically similar to \textit{u}, and extrapolate from their ratings of \textit{i}. \\
        Utility-based & Features of items in \textit{I}. & A utility function over items in \textit{I} that describes \textit{u}'s preferences. & Apply the function to the items and determine \textit{i}'s rank. \\
        Knowledge-based & Features of items in \textit{I}. Knowledge of how these items meet a user's needs. & A description of \textit{u}'s needs or interests. & Infer a match between \textit{i} and \textit{u}'s need. \\
        \\\hline
    \end{tabular}
    \label{table:burke02}
\end{table}

The properties by which Burke, 2002 \cite{Burke02}, classifies these systems are \textit{background data}, which exists prior to recommendation, \textit{input data}, the data that has been contributed by the user, and \textit{process}, the method by which recommendations are arrived at by using the \textit{background data} and \textit{input data}.

Burke, 2002 \cite{Burke02}, asserts that \textit{collaborative filtering} is probably the most widely used and mature of these types, citing \textit{GroupLens} \cite{Resnick94} and \textit{Tapestry} \cite{Goldberg92} as important examples of such systems, as well as several others. 

\textit{Collaborative filtering} a process whereby a user's preferences for items are inferred by the comparison of their previous preferences with the preferences of others. In itself this is a fairly straighforward principle, the comparison of two vectors of items and ratings or preferences, but there are a plurality of approaches. Burke, 2002 \cite{Burke02}, 

\textit{Content-based}, like \textit{collaborative filtering}, builds up a long term profile of a user's interests and preferences \cite{Burke02}.



\myparagraph{Hybrid Systems.}




--- off the web (?)
--- on the web...

-- What are the methods employed in recommender systems?

-- Collaborative Filtering
  - User-based filtering
  - Item-based filtering

-- Content-Based Filtering
  - Variants

-- Characteristics of the domain

--- Cold start problem
--- Sparsity problem
--- ... etc.

