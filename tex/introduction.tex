\iffalse 
Chapter 1: Introduction - the topic, the background, why the topic is relevant or of interest to you, what you hoped to achieve, the aims and objectives of the project.  
\fi

\section{Introduction}

\myparagraph{Recommender Systems.}

Since their origin in the mid-1990s with systems such as Tapestry \cite{Goldberg92} and GroupLens \cite{Resnick94}, recommender systems have become ubiquitous on the World Wide Web, being employed by some of the worlds largest online businesses as core parts of their offering to users.

Companies such as Amazon, Netflix, Facebook and Twitter use recommender systems to make all manner of suggestions to their users. These recommendations include such things as products, movies, news stories and other interesting users. 

It is the growth of the Web, which is now ubiquitous itself, that has given companies the ability to draw on unprecedented amounts of data about their users' preferences. At the same time the Web has made it easier than ever to reach their users with tailored suggestions.

Amazon's system of product recommendations using item-to-item collaborative filtering is regarded as a ``killer feature''\cite{Fortune12}, and is one of the defining features of the Amazon brand experience. Amazon state their mission to be, ``to delight our customers by allowing them to serendipitously discover great products''\cite{Fortune12}.

Netflix's movie recommender system \emph{Cinematch}  ``Netflix Prize'' competition 


\myparagraph{Social Recommender Systems.}

In recent years social applications have dominated the Web. Networks like Facebook and Twitter have become massive global businesses as Internet users share more and more of their lives online. In such a context recommender systems are able to look beyond users' relationships to product and services, being able instead to interrogate qualitative data about the relationships between specific people. Facebook describe this as the ``social graph''.

CITATIONS NEEDED!

These systems, ``a class of recommender systems that target the social media domain''\cite{Guy11}, represent the current state of the art. 

\myparagraph{Recommending Wines.}

Recommender systems for wines are not a new idea, being typical of the kind of item many systems are designed to recommend. Burke developed the VintageExchange FindMe recommender system in 1999\cite{Burke99}, and there is at least one patent pending with the WIPO for a wine recommender system\cite{WIPO12}.

As a knowledge-based recommender, Burke's FindMe system, ``required approximately one person-month of knowledge engineering effort''\cite{Burke99b}.

The Tetherless World Wine Agent (TWWA), by Patton and McGuinness\cite{Patton}. The TWWA project is primarily concerned with knowledge representaion and the Semantic Web, presenting a common and collaborative ontology for wine with which users can share wine recommendations across their social networks\cite{TWWAIndex}. The system does not automatically tailor recommendations to users, although this is stated as a target for future work\cite{TWWAIndex}.

\myparagraph{Service Oriented Architecture.}

I have chosen to implement my system as a service, such 

\myparagraph{Aims and Objectives.}

I aim to produce a recommender system for wines which takes advantage of both ratings and tasting notes to suggest the most interesting wines and users. I intend the system to ignore wine attributes, such as grape variety and colour, preferring to make recommendations based on a hybrid of pure collaborative filtering and a content-based filtering approach using user-submitted and expert tasting notes.

Rather than implement a full graphical interface for the system I have chosen to develop an HTTP API.

In doing so I will explore the field of recommender systems, 

Why are these systems interesting - Benefits - Challenges

Typical applications...  - Movies - Products (i.e. Amazon)

Applications in wine domain - what's the same - what's different

What will this project do?  - implement a recommender system for wines -
exploration of techniques etc.

