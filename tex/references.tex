\begin{thebibliography}{9}

    \bibitem{Burke02} Burke, \emph{Hybrid Recommender Systems: Survey and Experiments}, User Modeling and User-Adapted Interaction, Volume 12 Issue 4, November 2002, Pages 331 - 370. Kluwer Academic Publishers: Hingham, MA, USA

    \bibitem{Debnath08} Debnath, Souvik and Ganguly, Niloy and Mitra, Pabitra, \emph{
        Feature weighting in content based recommendation system using social network analysis}, Proceedings of the 17th international conference on World Wide Web, WWW '08, 2008, Beijing, China, Pages 1041 - 1042. ACM: New York, NY, USA,

    \bibitem{Goldberg92} Goldberg, D. Nichols, D., Oki, B. M., and Terry, D., \emph{Using collaborative filtering to weave an information tapestry}, Commun. ACM 35, 12 (Dec. 1992), 61--70.

    \bibitem{Fortune12} Mangalindan, J. P., \emph{Amazon's Recommendation Secret}, July 2012. URL: http://tech.fortune.cnn.com/2012/07/30/amazon-5/

    \bibitem{Resnick94} Resnick, P., Iacovou, N., Sushak, M., Bergstrom, P., Riedl, J., \emph{GroupLens: An open architecture for collaborative filtering of netnews}, 1994 ACM Conference on Computer Supported Collaborative Work, 1994. Association of Computing Machinery, Chapel Hill, NC.

    \bibitem{Resnick97} Resnick, P., Varian, H. R., \emph{Recommender Systems}, 1997. Communications of the ACM, 40 (3), 56-58. Association of Computing Machinery, Chapel Hill, NC.


\iffalse
A. Y. Ng and M. I. Jordan. On discriminative vs generative
classifiers: A comparison of logistic regression and naive
bayes. In Neural Information Processing Systems, pages
841–848, Vancouver, Canada, december 2001. MIT Press.
2
Robles, V.; Larranaga, P.; Menasalvas, E.; Perez, M.S.; Herves, V.; , "Improvement of naive Bayes collaborative filtering using interval estimation," Web Intelligence, 2003. WI 2003. Proceedings. IEEE/WIC International Conference on , vol., no., pp. 168- 174, 13-17 Oct. 2003
doi: 10.1109/WI.2003.1241189
keywords: {Algorithm design and analysis;Clustering algorithms;Collaboration;Collaborative work;Data mining;Filtering algorithms;Probability;Recommender systems;Scalability;Training data; Bayes methods; Web sites; groupware; information filters; learning (artificial intelligence); statistical analysis; Bayesian classifier; UCl repository; Web data; collaborative filtering; ecommerce site; interval estimation; naive Bayes method; recommender system; semi naive Bayes method;}
URL: http://ieeexplore.ieee.org/stamp/stamp.jsp?tp=&arnumber=1241189&isnumber=27823

Xiaoyuan Su; Greiner, R.; Khoshgoftaar, T.M.; Xingquan Zhu; , "Hybrid Collaborative Filtering Algorithms Using a Mixture of Experts," Web Intelligence, IEEE/WIC/ACM International Conference on , vol., no., pp.645-649, 2-5 Nov. 2007
doi: 10.1109/WI.2007.10
keywords: {Clustering algorithms;Collaborative work;Computer science;Filtering algorithms;International collaboration;Motion pictures;Niobium;Predictive models;Recommender systems;USA Councils;information filtering;content-boosted CF;experts mixture;hybrid collaborative filtering algorithms;memory-based algorithms;pure content-based CF algorithms;pure model-based algorithms;sequential mixture CF;}
URL: http://ieeexplore.ieee.org/stamp/stamp.jsp?tp=&arnumber=4427165&isnumber=4427044

Xiaoyuan Su; Taghi M. Khoshgoftaar; , "Collaborative Filtering for Multi-class Data Using Belief Nets Algorithms," Tools with Artificial Intelligence, 2006. ICTAI '06. 18th IEEE International Conference on , vol., no., pp.497-504, Nov. 2006
doi: 10.1109/ICTAI.2006.41
keywords: {Bayesian methods;Collaboration;Collaborative work;Filtering algorithms;Logistics;Predictive models;Recommender systems;Regression tree analysis;Robustness;Scalability;belief networks;data handling;groupware;information filtering;Bayesian belief nets;Pearson correlation-based collaborative filtering;data sparseness;extended logistic regression;multiclass collaborative filtering data;recommender system;tree augmented naive Bayes model;}
URL: http://ieeexplore.ieee.org/stamp/stamp.jsp?tp=&arnumber=4031936&isnumber=4031859



BibTex:

@conference {236,
title = {GroupLens: An open architecture for collaborative filtering of netnews},
booktitle = {1994 ACM Conference on Computer Supported Collaborative Work Conference},
year = {1994},
month = {10/1994},
pages = {175-186},
publisher = {Association of Computing Machinery},
organization = {Association of Computing Machinery},
address = {Chapel Hill, NC},
abstract = {<p class=``abstract''>Collaborative filters help people make choices based on the opinions of other people. GroupLens is a system for collaborative filtering of netnews, to help people find articles they will like in the huge stream of available articles. News reader clients display predicted scores and make it easy for users to rate articles after they read them. Rating servers, called Better Bit Bureaus, gather and disseminate the ratings. The rating servers predict scores based on the heuristic that people who agreed in the past will probably agree again. Users can protect their privacy by entering ratings under pseudonyms, without reducing the effectiveness of the score prediction. The entire architecture is open: alternative software for news clients and Better Bit Bureaus can be developed independently and can interoperate with the components we have developed.</p>},
doi = {http://doi.acm.org/10.1145/192844.192905},
author = {Resnick, P. and Iacovou, N. and Sushak, M. and Bergstrom, P. and J. Riedl}
}


\fi

\end{thebibliography}

